\chapter{\abstractname}

TPMs authenticate themselves by propagating their so-called endorsement certificate naming their manufacturer, who guarantees fulfillment of the TPM specification to ensure their security properties.
While this approach is sufficient for hardware TPMs as they are standalone chips, for firmware TPMs~(fTPM) it is assumed that the manufacturer of the fTPM is the same as that of all firmware components that were booted before the fTPM\@.
This is due to the fact that any previous firmware component can compromise the fTPM during loading.
The verifier in the remote attestation procedure has no way of verifying the entire boot chain up to the fTPM\@.
We propose a remote attestation system that provides the verifier with this possibility.
To this end, we utilize a new hardware root of trust that implements DICE\@.
The verifier only needs to trust the manufacturer of DICE, while everything beyond that is explicitly attested by passing their identities to the verifier.
In this way, the manufacturer of the fTPM and each preceding component can be independent.
Our system also binds the data of the fTPM to the identity of the fTPM, so that the data can no longer be accessed if the fTPM or a preceding firmware component is changed or potentially compromised.
To summarize, our system allows a remote verifier to detect potentially malicious changes to the firmware components up to the fTPM, and that the fTPM's data is only accessible as long as the identity of the fTPM does not change, making downgrade attacks and modifications of the fTPM less attractive to attackers.
Conversely, this ensures that the data is only accessible for precisely this fTPM and not for other fTPMs.
