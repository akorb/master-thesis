\chapter{\abstractname}

Zero Trust is a cybersecurity paradigm in which a network, e.g., an enterprise network, is considered compromised.
Therefore, each device of every service request must be verified before the request is served.
This is made possible by remote attestation, which is enabled by TPMs, for example.
For that, they authenticate themselves by propagating their so-called endorsement certificate naming their manufacturer, who guarantees they conform to the TPM specification to ensure their security properties.
While this approach is sufficient for hardware TPMs as they are standalone chips, for firmware TPMs~(fTPM) it is assumed that the manufacturer of the fTPM is the same as that of all firmware components that were booted before the fTPM\@.
This is due to the fact that any previous firmware component can compromise the fTPM during loading.
The verifier in the remote attestation procedure lacks the ability of verifying the entire boot chain up to the fTPM\@.
We propose a remote attestation system that provides the verifier with this possibility.
To compensate for the lack of a hardware root of trust of an fTPM compared to a hardware TPM, we introduce DICE as the hardware root of trust.
The verifier only needs to trust the manufacturer of DICE, while every firmware component beyond is explicitly attested by passing their identities to the verifier.
The three benefits of our solution are that (i) the manufacturer of the fTPM and its preceding firmware components can be independent of each other, (ii) detection of modification of fTPM by remote verifier, and (iii) protection of the fTPM's data-at-rest.
% Explain (ii)
DICE measures the first firmware component, which is then repeated up to the fTPM\@.
These measurements are forwarded to the remote verifier, which can then detect potentially malicious changes to each measured component.
% Explain (iii)
The fTPM's data-at-rest is protected by binding it to the identity of the fTPM\@.
This means that the data of an fTPM is only accessible to the fTPM that created it as long as its identity does not change, which makes downgrade attacks and changes to the fTPM less attractive to attackers.
