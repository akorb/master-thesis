% !TeX root = ../main.tex
% Add the above to each chapter to make compiling the PDF easier in some editors.

\chapter{Discussion}\label{chapter:discussion}

\section{Assessment of the fulfillment of requirements}

\subsection{Security requirements}

\subsection{Attestation process requirements}

% Definitions

\ac{TCG} defines as part of their Trusted Attestation Protocol \cite{tap} the requirements for an attestation process to provide assurance to a verifier that it is (i) accurate, (ii) interpretable, and (iii) attributable.

(i) Accurate attestation data represents the actual state of the device.
This includes freshness, i.e., the data is not replayed and does not represent an old, outdated state of the device.

(ii) Intuitively, the data must be interpretable by the verifier.
In other words, the verifier must be able to derive a decision about the trustworthiness of the prover based on the attestation data.

(iii) It must be possible to assign the attestation data to a specific device, i.e., it must be verifiable that the attestation data originates from the prover.

% Why they are reached

\section{Higher level protocols' compatibility}

\section{Missing privacy implications}

% An attacker could use the TCIs to find the exact version of the running software and match this to known vulnerabilities of this version.
% See: https://www.rfc-editor.org/rfc/rfc9334.html#section-11

\section{Hardware requirements DICE + fTPM vs TPM}

% Does it even make sense? (Price wise, maybe energy wise)

\section{Personal opinion about developed system}

% Maybe too complex?
% Bad feeling about this?
% Benefit bigger than effort?
