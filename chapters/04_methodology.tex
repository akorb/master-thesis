% !TeX root = ../main.tex
% Add the above to each chapter to make compiling the PDF easier in some editors.

\chapter{Methodology}\label{chapter:methodology}

\section{Terminology}
\label{sec:terminology}

Before we dive into technical explanations, we want to clear some potential terminology confusion.

In the original DICE release from Microsoft \cite{England2016}, the identifier of a component is called the Firmware ID (FWID).
The TCG consortium later renamed it TCB Component Identifier (TCI).
We believe this is to emphasize that the TCI does not necessarily have to be the hash of a firmware binary, but could also be, for example, the embedded ID of a hardware component.
However, TCG has not fully implemented this terminology renaming.
Their DICE Attestation Architecture \cite{TCGAttestation2020} defines an X.509 extension that contains the TCIs.
They continue to be referred to as FWIDs in the formal definition of this extension, while everywhere else they are referred to as TCIs.
In personal correspondence with TCG, we have learned that this is due to backwards compatibility.
The old term FWID is retained whenever it is used in something that is alive in the long term, like formal definitions, and the new term TCI in assets that can be updated more quickly, such as the specification text.
Therefore, we will use the term TCI in this theoretical chapter, and in the implementation chapter (\autoref{chapter:implementation}) we will use the term FWID, just as it is common practice at the TCG.

% Prover/Verifier (in many earlier papers) vs. Attester/Verifier/Relying Party (RFC)

%\section{The identities of an fTPM}

% chain identity
% component identity

\begin{figure}[htpb]
  \centering
  \includegraphics[width=0.6\linewidth]{figures/identities.pdf}
  \caption{The \ac{CDI} and the \ac{TCI} from the perspective of an \ac{fTPM} component with preceding firmware~(FW). The \ac{UDS} provides the hardware's identity.}\label{fig:identities}
\end{figure}


% TODO: Update this

Depending on the point of view, there are two identities of an fTPM, as shown in \autoref{fig:identities}.
There is the identity of the fTPM measured by DICE, which consists of the hash of the binary and the component configurations as for all DICE components, as shown in \autoref{fig:ftpm-identity}.
This identity is referred to as the TCI.
Compilation flags are not part of these configurations, as they are embedded in the final binary and are therefore automatically measured as part of the measurement of the binary.
Of more interest are the configurations that are not part of the binary.
They are usually provided in well-known formats such as \texttt{json} or \texttt{xml}.
However, Microsoft's fTPM reference implementation does not contain such configurations, which simplifies the TCI generation of our fTPM by limiting it to the measurement of the fTPM binary data.
The TPM's storage cannot be part of its identity as it changes during runtime after each data write, e.g., storing an arbitrary key.
This is a problem because DICE only runs during the boot time in which the identity of the fTPM is measured.
The identity of the fTPM must not change afterwards, otherwise the consistency of the identity transmitted by DICE and the actual identity of the fTPM would differ.
In the interest of completeness, we also do not want to restrict the permissible values of the working data of an fTPM.
In summary, the CDI keeps being updated from layer to layer, the TCI is calculated for each layer individually.
Then, there is also the identity of the fTPM in the TPM context, which is represented by its EK.
This identity is not only bound to the binary and the configuration of the fTPM, but also the entire underlying firmware stack.
We derive the EK from the TPMs CDI, which is derived of the measurements of all preceding firmware components and the TPMs' TCI, see \autoref{eq:dice_cdi}.

\begin{figure}[htpb]
  \centering
  \includegraphics[width=0.6\linewidth]{figures/ftpm-identity.pdf}
  \caption{The accumulation of the binary and its configuration to its identity.} \label{fig:ftpm-identity}
\end{figure}


% Configurations required
% Does contain whole firmware stack?
% Data cannot be considered, since changes during runtime, not represented by DICE which only happens at boot-time
% What happens if identity changed? -> Data not accessible, in essence, fTPM fully reset
% https://trustedfirmware-a.readthedocs.io/en/latest/design_documents/measured_boot.html#critical-data


\section{Architectural overview}

\begin{figure}[htpb]
  \centering
  \includegraphics[width=1\linewidth]{figures/architecture.pdf}
  \caption{The architecture of our system.} \label{fig:architecture}
\end{figure}


The architecture of our proposed and later implemented system is illustrated in \autoref{fig:architecture}.
As you might notice, it is similar to our overview picture of DICE (\autoref{fig:dice-layers}).
This is to be expected, since our system uses DICE.
We leverage the DICE as our static root of trust for measurements (S-RTM).
Static in this context means that it uses the trusted state that a device has at the always same point in time, here after switching on, for further measurements.
This is in contrast to a dynamic RTM (D-RTM), which is able to do this at any time, e.g., Intel SGX.

The boot process continues from here in the usual DICE manner until the firmware TPM is reached.
The component that measures the fTPM is usually the trusted operating system running in EL1 in the secure world, as seen in \autoref{fig:arm_trustzone_arch}.
Like any other DICE component, the fTPM receives its secret Compound Device Identifier (CDI) from its predecessor layer.
Recall that the CDI is tied to the identity of the fTPM including the entire underlying firmware stack.
Two values are derived from the CDI.

% Storage Key generation

First, the storage key is generated.
This is a symmetric encryption key that is used to encrypt the fTPM storage space in RAM before it is written to a persistent storage space such as a hard disk drive (HDD).
At no time does the HDD see plain text data.
Since the storage key is derived from the CDI, the old storage data is inaccessible if the identity of the TPM changes, e.g. due to a TPM modification or an update of a previous firmware component, which is equivalent to a full manufacturer reset.
This enables the property that an fTPM memory must never be accessible to another TPM.

\begin{figure}[htpb]
  \centering
  \includegraphics[width=0.8\linewidth]{figures/storage-encryption.pdf}
  \caption{The fTPMs' storage is protected by a key derived from its identity.}\label{fig:storage-encryption}
\end{figure}


% EPS Generation

Then, the primary endorsement seed (EPS) is generated.
It is the seed that is used to generate the primary endorsement key (EK).
A primary key in the sense of the TPM means that it has no parent key, but a parent seed, here the EPS.
The indirection of generating the CDI via the EPS instead of generating the EK directly from the CDI is introduced because the code of fTPMs can be hardcoded to use the EPS during EK generation.
And we want our system to require as few modifications to TPM code as possible.

% Device identification

With a dedicated TPM, the EK can be used to represent the long-term identity of the device as long as the TPM is not soldered away.
Our EK does not do this because a firmware TPM software-based and changes every time the fTPM or the underlying firmware is modified.
Instead, we use the DICE for this, which is hardware-based.
Its DeviceID key, as the name suggests, represents the device identity.
Note that the DeviceID contains the identity of layer 0 of the boot chain, i.e., the first mutable code.
This can also be seen in \autoref{eq:dice_deviceID}.
For this reason, the DICE specification suggests keeping the first mutable code as small as possible so that it remains constant throughout the life of the device \cite{dice-layering-arch}.

% EK = AK

Our EK takes on the role of an attestation key.
In general, attestation keys represent device identities and are therefore privacy-sensitive.
Although our EK does not represent device identities, it does represent the identity of short-lived fTPMs, which is also privacy-sensitive.
Our approach has the advantage that we do not need a trusted third party, but trust the producer of our trust anchor directly (DICE).
In \autoref{sec:privacy}, an extension of our system is presented with privacy in mind.

% 5.3.2 from https://dl.acm.org/doi/10.1145/3098954.3103165
% Use the references there to explain why we used HMAC with SHA256
% Even though https://trustedcomputinggroup.org/wp-content/uploads/Hardware-Requirements-for-Device-Identifier-Composition-Engine-r78_For-Publication.pdf says hash-only is also an option, https://nvlpubs.nist.gov/nistpubs/SpecialPublications/NIST.SP.800-57pt1r5.pdf (Table 3) shows that the HMAC version has roughly double security strength. Cite all three here (ACM, TCG, NIST)!




% \section{Provisioning process}

% To summarize, attestation key provisioning must ensure that only valid attestation key material is established in Attesters [RFC 9334]

\section{Attestation process}

% Check also for revocation of manufacturer cert

\begin{equation}
trusted(C_{i}) \, \coloneqq \, \bigwedge_{k=0}^{i} trusted(C_{k})
\end{equation}

% We use explicit attestation. See DICE Layering Spec 7.2
% See 8.1.1.2 for an example about implicit attestation
% Implicit attestation would require to know all resulting public keys in EKcert, which would require that we booted it up in a controlled manner beforehand and stored the resulting code.
We use an explicit attestation procedure.
This makes it sufficient for the verifier to know the trusted TCI, whereas implicit attestation would require a database of known public keys corresponding to a trusted TCI.
And since each public key for a device is unique, as it is ultimately derived from the device's unique UDS, the verifier would need to know the mapping between public keys and the corresponding TCI for each device, which we consider unrealistic.
Also, this is a hindrance as the verifier should be able to trust an unknown device by trusting the DICE manufacturer.

\section{Updating the fTPM}

We consider it as critical that the \ac{fTPM} is updatable. This is due to the history of \acp{fTPM} showing vulnerabilities which have been patched consequently. % cite... all from background probably

Our \ac{fTPM} can be only updated with the system shut down. This is due to the required out-of-band signing procedure of trusted applications before being deployed. This also  with while system is shut down. This ensures that the TCI part of the EKcert generated at boot-time does not become obsolete, in other words, keeps representing the state of the currently running fTPM.

% Clear data

To protect against downgrade attacks:
NV data is encrypted/integrity with AESP
Encryption required to ensure confidentiality
AESP used to also ensure integrity
This is required since also the cipher text of the NV could be modified, which might change security critical information.
Encryption-only would not prohibit that.
As soon as an integrity violation is detected, the \ac{fTPM} is fully reset, effectively invalidating all previously stored data.
Note that an attacker could thereby easily trigger a data loss.
This has to avoided by integrating the good-practices with working with a \ac{TPM}, which includes having secrets stored also elsewhere. % Maybe cite something, I feel like Microsoft has something for that
This introduces storage and memory overhead.
Processing overhead only slightly, since the data is already decrypted during start-time, which happens only once at boot time, and then later data is encrypted only while it is stored, which happens only ... % TODO: When s it happen? I remember there were not many places in the code where storing of the NV onto the persistent storage is triggered. 
Hence, there is no performance penalty during common uses of a \ac{TPM}, e.g., key creation.

% Add layered figure which shows that the hard drive never sees decrypted data, but only the secure memory. Important because data can be stored in NW.
This might seem redundant because of the storage protection of for example TrustZone, but this does not protect against downgrade attacks. With our approach, the access to the data is bound to the exact identity of the fTPM including all underlying firmware.

So, our protection additionally protects data-at-rest, while the data-at-use is protected by the TEE's secure memory, i.e., the memory isolation from the normal world.



\section{Privacy}
\label{sec:privacy}

% General: Contradiction
% Goal of DICE is explicitly to establish a long-term identity. So, we need to work against its design.

% Maybe: The certificates would need to be encrypted as well, as they contain the TCI values, potentially identifying the device

% Use Attestation key not in endorsement (or platform?) hierarchy

% Each AliasKey is a identity key as specified in DICE layering Architecture 8.1.1.5
% They would need to be made independent of the devices identity.
