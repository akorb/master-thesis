% !TeX root = ../main.tex
% Add the above to each chapter to make compiling the PDF easier in some editors.

\chapter{Related Work}\label{chapter:related_work}

% Other defense concepts

In the following, we describe defense mechanisms for fTPMs that can be seen as complementary to our approach. They all have in common that they offer no way for a third party to ensure that the hardened fTPM is actually running on the device under test, which is exactly what our work aims to cover.

One approach is to verify the code of fTPMs \cite{Mukhamedov2013}. Here, the TPM 1.2 code is written in a functional programming language that enables automatic verification.

There exist efforts to improve the security of TPM by introducing the concept of hybrid TPMs \cite{Kim2019, Gross2021}. Kim and Kim \cite{Kim2019} extend a hardware TPM with software support, which they name hTPM. This increases the defense of the TPM, e.g., circumventing side-channel attacks, and also enables more secure TPM functions, e.g., enabling true random number generation. Their hTPM implementation also shows significantly better performance due to the use of modern CPU features.
Vice versa, Gross et al. \cite{Gross2021} propose the reverse approach of backing an fTPM with hardware. While their implementation has similar properties to hTPM, it inherits some downsides of fTPMs. For example, their fTPM is still started later in the boot chain than a dTPM, which is not the case for hTPM. However, it is easier to update than hTPM since the lack of a dTPM, and the overall design is simpler.

% Attacks which we would avoid (e.g., exchange/spoof EKCert)
