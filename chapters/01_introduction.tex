% !TeX root = ../main.tex
% Add the above to each chapter to make compiling the PDF easier in some editors.

\chapter{Introduction}\label{chapter:introduction}

This chapter includes an explanation of the exact problem we are addressing, and why, a brief overview of our solution, and the attacks we are trying to fend off.

\section{Motivation}

Modern trust relationships, such as Zero Trust \cite{isaca2021}, require trustworthy platforms, which can reliably report their system state.
In such models, trustworthiness can only be assumed after the platform configuration has been proved by all parties of the communication.

In the meantime, TPMs rise in their deployments and importance, e.g., in 2013 the President's Council of Advisors on Science and Technology encourages the adoption of TPMs \cite{usa}, and Microsoft publicized that they require a TPM module for Windows~11 in 2021 \cite{win11req}.

A dedicated trusted hardware TPM, which would allow for independent attestation of system states, increases cost and complexity - especially for embedded platforms.
Integrated solutions for resource-constrained devices such as \ac{DICE} are unsuitable for large dynamic systems, for example Linux based devices.
These mechanisms shift trust from the firmware provider to the hardware provider by allowing firmware attestation through a hardware root of trust.
Through trusted execution environments (TEEs), such as Arm TrustZone, a firmware-TPM (fTPM) can be used to provide similar security guarantees as a hardware TPM chip.
This approach is currently very limited, as no comprehensive mechanisms exists to integrate a fTPM into DICE-like attestation mechanisms.
Without such mechanisms, independent verifiers have to blindly trust the firmware manufacturer, which drastically limits trust relationships.

An independently verifiable fTPM, rooted in a hardware trust anchor, can be leveraged in a
zero trust environment without requiring additional hardware or compromising on security.

\subsection{Concept}
The conceptual basis for this feature is to attest the software stack of the TPM itself, and thus providing a way to independently assess the properties of the fTPM.
As the fTPM typically runs atop different software components itself, they too have to be included in the attestation of the fTPM.
Current fTPM implementations require additional security measures to not leak state between reboots and different software versions.
The final concept should provide a comprehensive guideline to implement an fTPM, which accounts for such an environment and reflect any relevant information through remote attestation.

\section{Goal}
\section{Threat Model}

The main threat is the modification of the binary of the fTPM before or during boot. For example, by exchanging the SD card storing the binary.
However, we assume that the fTPM cannot be modified by malicious parties after the boot process (regardless of whether the fTPM is benign or compromised) because we trust the OP-TEE environment.
Out-of-scope are hardware attacks, side-channel attacks, control-flow attacks, and Denial of Service attacks.

% - In-scope
%     - Modification of fTPM (at boot, not runtime) (Q: Only at boot-time because the OP-TEE is trusted and therefore, no (malicious) runtime modifications are expected, correct?)
%     - Exchange of SD card
% - Out-of-scope
%     - HW attacks
%     - Side-channel attacks
%     - Control-flow attacks
%     - DoS attacks


\section{Environment}

This work was created at the 'Fraunhofer-Institut für Angewandte und Integrierte Sicherheit AISEC' in Garching.
It is part of the 'Fraunhofer Society for the Promotion of Applied Research e.~V.', which is an organization distributed over Europe with main focus on applied research.
In the roughly 35 years of its existence, it rose to become the largest research institute in Europe with around 30,000 employees.

\section{Outline}
