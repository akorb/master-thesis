% !TeX root = ../main.tex
% Add the above to each chapter to make compiling the PDF easier in some editors.

\chapter{Introduction}\label{chapter:introduction}

\section{Motivation}

TPMs rise in their deployments and importance, e.g., in 2013 the President's Council of Advisors on Science and Technology encourages the adoption of TPMs \cite{usa}, and Microsoft publicized that they require a TPM module for Windows~11 in 2021 \cite{win11req}.

% zero trust \cite{isaca2021}

\section{Goal}
\section{Threat Model}

The main threat is the modification of the binary of the fTPM before or during boot. For example, by exchanging the SD card storing the binary.
However, we assume that the fTPM cannot be modified by malicious parties after the boot process (regardless of whether the fTPM is benign or compromised) because we trust the OP-TEE environment.
Out-of-scope are hardware attacks, side-channel attacks, control-flow attacks, and Denial of Service attacks.

% - In-scope
%     - Modification of fTPM (at boot, not runtime) (Q: Only at boot-time because the OP-TEE is trusted and therefore, no (malicious) runtime modifications are expected, correct?)
%     - Exchange of SD card
% - Out-of-scope
%     - HW attacks
%     - Side-channel attacks
%     - Control-flow attacks
%     - DoS attacks


\section{Environment}

This work was created at the 'Fraunhofer-Institut für Angewandte und Integrierte Sicherheit AISEC' in Garching.
It is part of the 'Fraunhofer Society for the Promotion of Applied Research e.~V.', which is an organization distributed over Europe with main focus on applied research.
In the roughly 35 years of its existence, it rose to become the largest research institute in Europe with around 30,000 employees.

\section{Outline}
