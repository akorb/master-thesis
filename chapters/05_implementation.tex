% !TeX root = ../main.tex
% Add the above to each chapter to make compiling the PDF easier in some editors.

\chapter{Implementation}\label{chapter:implementation}

As explained in \autoref{sec:terminology}, from now on the term Firmware ID (FWID) will be used instead of the previously used term TCB Component Identifier (TCI).


% Use the attributes for the NVindex of the EKcert as defined by TPM PC Client Spec 4.5.2.1
% The attributes set such that the NV index can only be written or deleted if the policy is fulfilled. However, it has an empty policy, which can never be fulfilled. This yields an undeletable EK certificate.

% \section{Providing a custom EK template}
\section{Adaption of the Endorsement Key}

The EK is a primary key, so it is derived from the Endorsement Primary Seed (EPS).
There is also a default template defined by TCG~\cite{tcg-ek}, dictating the endorsement key to be a restricted encryption key, since it is privacy-sensitive.
The default template is required to be able to reproduce the EK contained in the EK certificate so that the TPM can prove that this EK certificate corresponds to it by being able to generate the corresponding private key.
The default template is not stored on the TPM, but it is part of the command triggering the key generation (\texttt{TPM2\_CreatePrimary}).
Therefore, the TPM itself does not need to know what the default template is, since it is always provided by TPM-capable software.

However, we want to use the EK as a signing key.
The TPM specification provides a mechanism for this, where we need to store our custom EK template in a specific NV index within the TPM~\cite{tcg-ek}.
We use the values of the default EK template and only deviate from it when necessary in three aspects.
We declare the EK as a (i) restricted signing key.
This also requires to specify a (ii) signature scheme, and (iii) no inner symmetric key as required for signing keys since they are not allowed to have any children keys.

This template will be generated within the TPM after each manufacturer reset, so it will be preserved even after an identity change and a reset of the firmware TPM\@.
However, the EK itself will change as the CDI changes, then the EPS and finally the EK\@.
The attributes of the NV index are declared such that the template cannot be deleted~\cite{tcgPcClient}.
This is possible by allowing to delete the NV index only when an unfulfillable policy is met.

% Ensure that the template generated by code is also directly used, instead of indirectly via unmeasured storages.
The NV storage is not attested since it's ``working data'' and not ``configuration data'', and would be hard to attest since it's encrypted and the cipher text often changes since the IV is generated randomly on each store.
However, the component right before the fTPM knows the fTPM's CDI, could generate the according symmetric key to decrypt the NV storage, and integrate the required NV values in the TCI\@.
By baking the EK template generation in code which is already attested (and the code exists anyways), we prevent this complexity.

The ``working data'' is not attested, since this would restrict the functionality of the fTPM\@.
So, the an NV index is with ordinary TPM commands, providing the required policy/authentication is fulfilled.
We consider this configuration as out-of-scope for attestation, for the former mentioned reason.
In other words, it could be that everyone can change the NV index, where the template is stored.
And in the subsequent boot, we would create a EKcert for it, without anything.
Therefore, during EKcert creation, generate the template in code (which is attested), and use this directly. 

% 5.3.2 from https://dl.acm.org/doi/10.1145/3098954.3103165
% Use the references there to explain why we used HMAC with SHA256
% Even though https://trustedcomputinggroup.org/wp-content/uploads/Hardware-Requirements-for-Device-Identifier-Composition-Engine-r78_For-Publication.pdf says hash-only is also an option, https://nvlpubs.nist.gov/nistpubs/SpecialPublications/NIST.SP.800-57pt1r5.pdf (Table 3) shows that the HMAC version has roughly double security strength. Cite all three here (ACM, TCG, NIST)!

\section{Prover}
\subsection{Normal World}

\subsection{Secure World}

\subsubsection{Measuring the fTPM}

\section{Attester}

\section{Times}

Usually, the notBefore time of Alias certs should be build time, and notAfter should be infinite.
Network-enable our FVP, but heavy setup for simple demonstration of our system.
Could malfunction sometimes because of time dependencies.
To keep our system easily understandable and simple to access a demonstration, we use fixed times.

% List datetimes of all certificates and FVP system time
% Maybe in a graph with timeline

\section{Technical obstacles}

We would have liked to use RSASSA-PSS which is formally proven to be secure over RSASSA-PKCS1-v1\_5.
RFC 8017 even requires RSASSA-PSS for new applications~\cite{Moriarty2016}.
However, it is not fully supported by the tpm2-tools, yet\footnote{\url{https://github.com/tpm2-software/tpm2-tools/issues/3283}}.
