% !TeX root = ../main.tex
% Add the above to each chapter to make compiling the PDF easier in some editors.

\chapter{Future Work and Conclusion}\label{chapter:future_work_and_conclusion}

\section{Future Work}

The logical consequence is the implementation of our solution on real hardware instead of in a simulation environment such as FVP\@.
This allows the interaction with the hardware to be verified, especially with an RPMB partition that requires hardware support.
In addition, the impact of our solution on the performance of the system can then be measured in practice.
This is especially interesting since we integrate another storage encryption.

Although it makes sense to show our solution on Arm hardware first, we are, as mentioned, not limited to it.
Therefore, a future work is to concretize the description of our implementation for TEE technologies other than Arm, e.g. Intel~SGX\@.

The system we propose can also be transferred from the DICE architecture to other technologies that also perform firmware measurements.
A new technological framework that generalizes DICE is Caliptra~\cite{caliptra}.
It is based on the concept of DICE, but is not limited to it.

As mentioned in \autoref{sec:arch_overview}, RPMB's rollback protection only protects against attacks from outside the TEE\@.
We would like to establish a design that tightens the rollback protection from the trust of the entire TEE to the identity of the fTPM\@.
This can probably be achieved by encrypting the metadata stored on the RPMB with the storage key derived from the identity of the fTPM, i.e., its CDI, before sending it to the RPMB\@.
This must be implemented in the trusted operating system and not in the fTPM TA, as the trusted operating system normally manages the metadata.

Furthermore, our solution does not protect against runtime attacks on the fTPM\@.
In general, trusted applications in a TEE are not resistant to security problems caused by programming errors, e.g., buffer overflow attacks.
Therefore, remote attestation of the control flow integrity of the fTPM may be a desired function.
Displaying the current state of the fTPM instead of its identity at boot time would be a useful extension to our solution.

\section{Conclusion}

In this work, we have proposed a novel remote attestation scheme to establish trust in a firmware TPM\@.
fTPMs cannot be trusted based on their isolated identity alone, as their underlying software components are also security relevant, unlike dedicated TPMs which are a separate chip.
We therefore use the DICE as a hardware root of trust and measure each component during the boot process up to the fTPM\@.
The verifier can thus learn the measurements of the corresponding components and decide whether they are classified as trustworthy.
These measurements are transmitted from the prover to the verifier in the form of certificates.
Since it is in the nature of certificates that they are not considered secret and can therefore be easily replayed by malicious provers, the verifier must ensure that the certificates correspond to the device he is communicating with.
This is not directly ensured by our system, but should be part of the attestation protocol that runs atop of our system, e.g., the fTPM, which attests the state of the system with a quote.
To do this, the prover must prove that he has the private key that corresponds to the public key part of the certificate that describes the firmware TPM\@.
These keys are unique for the identity of the device (the \ac{UDS}) and the identities of the individual components (the TCIs) and therefore cannot be generated by other, potentially malicious, verifiers.
We concluded the presentation of our system with an explanation of a proof-of-concept implementation and a discussion of the feasibility, caveats and limitations of the system.

We believe that our system is an important step towards the independence of the manufacturer of the firmware TPM and its upstream software, whereas today provers trust a single manufacturer who is assumed to have provided all these components.
Our system also creates a hardware root of trust for the firmware TPM which, as the name suggests, cannot be provided by the fTPM as it is a software component.
In contrast, dedicated TPMs are capable of acting as a hardware root of trust.
Our system closes this gap.

% TODO: Write about removing 'TA_FTPM_ATTEST' and instead store the certificate chain in the NV indices provided by EK profile.
% We didn't do it because it does not benefit the demonstration of our system, but requires much engineering effort. For example, tpm2_getekcertificate does not support that.
% But it would greatly simplify our solution. But it also takes NV index storage space, which is now not taken. But the fTPM still takes more space.
