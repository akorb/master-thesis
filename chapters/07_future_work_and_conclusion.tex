% !TeX root = ../main.tex
% Add the above to each chapter to make compiling the PDF easier in some editors.

\chapter{Future Work and Conclusion}\label{chapter:future_work_and_conclusion}

\section{Future Work}

\section{Conclusion}

In this work, we have proposed a novel remote attestation scheme to establish trust in a firmware TPM\@.
fTPMs cannot be trusted based on their isolated identity alone, as their underlying software components are also security relevant, unlike dedicated TPMs which are a separate chip.
We therefore use the DICE as a hardware root of trust and measure each component during the boot process up to the fTPM\@.
The verifier can thus learn the measurements of the corresponding components and decide whether they are classified as trustworthy.
These measurements are transmitted from the prover to the verifier in the form of certificates.
Since it is in the nature of certificates that they are not considered secret and can therefore be easily replayed by malicious provers, the verifier must ensure that the certificates correspond to the device he is communicating with.
This is not directly ensured by our system, but should be part of the attestation protocol that runs atop of our system, e.g., the fTPM, which attests the state of the system with a quote.
To do this, the prover must prove that he has the private key that corresponds to the public key part of the certificate that describes the firmware TPM\@.
These keys are unique for the identity of the device (the UDS) and the identities of the individual components (the TCIs) and therefore cannot be generated by other, potentially malicious, verifiers.
We concluded the presentation of our system with an explanation of a proof-of-concept implementation and a discussion of the feasibility, caveats and limitations of the system.

We believe that our system is an important step towards the independence of the manufacturer of the firmware TPM and its upstream software, whereas today provers trust a single manufacturer who is assumed to have provided all these components.
Our system also creates a hardware root of trust for the firmware TPM which, as the name suggests, cannot be provided by the fTPM as it is a software component.
In contrast, dedicated TPMs are capable of acting as a hardware root of trust.
Our system closes this gap.
